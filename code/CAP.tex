\documentclass[12pt]{article}
\usepackage{longtable}
\usepackage{geometry}
\geometry{left=1.25 in,right=1.25 in,top=1.00in,bottom=1.0in}

%\documentclass[justified,notoc]{tufte-handout2}
%need to figure out why figure refernces are messed up

\usepackage{natbib}  % call natbib
\setcitestyle{authoryear} % set citation style to authoryear
\bibliographystyle{plainnat} % use the plainnat instead of plain
\usepackage{booktabs}
\usepackage{amssymb, amsmath, amsfonts}
\usepackage{outlines}
\usepackage{soul}
\usepackage[gen]{eurosym}
\usepackage{graphicx}
\usepackage{fancyhdr}
\usepackage{authblk}
\usepackage{lineno}
\usepackage[hyphens]{url}
\usepackage{hyperref}
\hypersetup{
 colorlinks = true, %Colours links instead of ugly boxes
 urlcolor = blue, %Colour for external hyperlinks
 linkcolor = blue, %Colour of internal links
 citecolor = blue %Colour of citations
}
\pagestyle{fancy}
\usepackage{outlines}
\usepackage{caption}
%\captionsetup[table]{name=Figure}
\graphicspath{{../input/}}
\usepackage[outdir=./]{epstopdf}


\usepackage{enumitem}
\setlist[enumerate,2]{label=\roman*)}
\setlist[enumerate,3]{label=\Roman*)}
\setlist[enumerate,4]{label=\roman*)}

%\renewcommand{\familydefault}{\sfdefault}
% ------------------------------- use \citep{FAQs} to cite
\usepackage{filecontents}
\begin{filecontents}{\jobname.bib}
}

@article{Ross2016a,
 title  = {{The Supervisory Capital Assessment Program}},
 publisher = "Yale Program on Financial Stability",
 author = "Ross, Chase P.",
 year  = "2016",
 doi  = " ",
 volume = " ",
 journal = "Yale Program on Financial Stability Intervention Case",
 issn  = " ",
 url  = {http://papers.ssrn.com/sol3/papers.cfm?abstract_id=2722712},
}

@article{Ross2016b,
 title  = {{The Capital Purchase Program}},
 publisher = "Yale Program on Financial Stability",
 author = "Ross, Chase P.",
 year  = "2016",
 doi  = " ",
 volume = " ",
 journal = "Yale Program on Financial Stability Intervention Case",
 issn  = " ",
}


@article{SCAPResults,
 title  = {{The Supervisory Capital Assessment Program: Overview of Results}},
 publisher = "Board of Governors of the Federal Reserve System",
 author = {{Federal Reserve}},
 year  = "2009",
 doi  = " ",
 volume = " ",
 journal = " ",
 issn  = " ",
 url  = {http://www.federalreserve.gov/newsevents/press/bcreg/bcreg20090507a1.pdf},
}

@article{Caballero,
 title  = {{A (Mostly) Private Capital Assistance Programme (CAP)}},
 publisher = " ",
 author = "Caballero, Ricardo",
 year  = "2009",
 doi  = " ",
 volume = " ",
 journal = "VoxEU",
 issn  = " ",
 url  = {http://voxeu.org/article/recapitalising-banks-caballero-plan},
}

@book{Geithner,
 title={{Stress Test: Reflections on Financial Crises}},
 author={Geithner, Timothy F},
 year={2014},
 publisher={Crown}
}

@misc{Paulson,
 title={{October 14, 2008 Speech}},
 author={Paulson, Hank},
 year={2008},
  url  = {http://ftalphaville.ft.com/2008/10/14/17036/paulsons-nine-strong-posse/},
}

@misc{CPPInstructions,
 title={{Application Guidelines for TARP Capital Purchase Program}},
 author={Treasury},
 year={2008},
  url  = {https://www.treasury.gov/initiatives/financial-stability/TARP-Programs/bank-investment-programs/cap/Documents/application-guidelines.pdf 
},
}

@book{Bernanke,
 title={{The Courage to Act: A Memoir of a Crisis and its Aftermath}},
 author={Bernanke, Ben S},
 year={2015},
 publisher={WW Norton \& Company}
}

@article{CAPTerms,
 title  = {{U.S. Treasury Releases Terms of Capital Assistance Program}},
 author = "U.S. Treasury",
 year  = "2009",
 doi  = " ",
 volume = " ",
 journal = " ",
 issn  = " ",
 url  = {https://www.treasury.gov/press-center/press-releases/Pages/tg40.aspx},
}

@article{WhitePaper,
 title  = {{The Capital Assistance Program and its Role in the Financial Stability Plan}},
 author = "U.S. Treasury",
 year  = "2009",
 doi  = " ",
 volume = " ",
 journal = " ",
 issn  = " ",
 url  = {https://www.treasury.gov/press-center/press-releases/Documents/tg40_capwhitepaper.pdf},
}

@article{GW,
 title={{Valuing the Treasury's Capital Assistance Program}},
 author={Glasserman, Paul and Wang, Zhenyu},
 journal={Management Science},
 volume={57},
 number={7},
 pages={1195--1211},
 year={2011},
url={https://www.newyorkfed.org/research/staff_reports/sr413.html},
}

@article{Mofo,
 title={{Capital Assistance Program Public (CAP) Cheat Sheet}},
 author={{Morrison Foerster}},
 year={2009},
url  = {http://media.mofo.com/docs/pdf/090310CAPPublicCheatSheet.pdf},
}

@article{OFS,
 title={{Troubled Asset Relief Program: Two Year Retrospective}},
 author={{Office of Financial Stability}},
 year={2010},
url  = {https://www.treasury.gov/press-center/news/Documents/TARP\%20Two\%20Year\%20Retrospective_10\%2005\%2010_transmittal\%20letter.pdf},
}

@article{Citizens,
 title={{Citizens Republic Bancorp Withdraws Its CAP Application}},
 author={{PR Newswire}},
 year={2009},
}

@article{Midwest,
 title={{Midwest Banc Holdings, Inc. Reports Q3 Results}},
 author={{PR Newswire}},
 year={2009},
}

@article{CPPFaq,
 title  = {{Process-Related FAQs for Capital Purchase Program}},
 author = "U.S. Treasury",
 year  = "2008",
 doi  = " ",
 volume = " ",
 journal = " ",
 issn  = " ",
 url  = {https://www.treasury.gov/press-center/press-releases/Documents/faqcpp.pdf},
}

@article{Dash,
 title={{Citigroup Sheds Energy Unit and Its \$100 Million Trader}},
 author={Dash, Eric and Healy, Jack},
 year={2009},
 publisher={The New York Times},
 url={http://www.nytimes.com/2009/10/10/business/10citi.html?_r=1},
}

@article{Andrews,
 title={{U.S. May Convert Banks' Bailouts to Equity Share}},
 author={Andrews, Edmund L.},
 year={2009},
 publisher={The New York Times},
 url={http://www.nytimes.com/2009/04/20/business/20bailout.html?_r=1},
}

\end{filecontents}
% -------------------------------

\begin{document}

	\lhead{}
	\rhead{}
	\renewcommand{\headrulewidth}{0.0pt}
	\renewcommand{\footrulewidth}{0.0pt}

\title{The Capital Assistance Program}
\author{Chase P. Ross\thanks{\texttt{\href{mailto:chase.ross@yale.edu}{chase.ross@yale.edu}}}}
\affil{Yale Program on Financial Stability \\ Yale School of Management}
\date{}


\maketitle

\begin{abstract}
On February 10, 2009 U.S. financial regulators announced a two-part financial stability plan in order to revive confidence in the U.S. banking system and restart bank lending. The first phase of the plan was a stress test called the Supervisory Capital Assessment Program (SCAP) to assess the capital adequacy and needs of the 19 largest bank holding companies if economic conditions deteriorated further. The second phase of the plan, the Capital Assistance Program (CAP) was a public capital backstop to the banking system available to banks which were deemed to have insufficient capital to absorb losses if the economic conditions deteriorate further and unable to raise capital from the private sector. Ultimately, the Treasury did not receive any applications for the CAP, and the program closed on November 9, 2009.
\newline
\newline
\textbf{JEL Classification}: G01, G28, G20, H12, H81
\newline
\textbf{Keywords}: crisis intervention, Capital Assistance Program, TARP, capital requirements, systemic risk, Tier 1 Capital

\end{abstract}
\newpage
\tableofcontents

\newpage
\linenumbers
\section{Overview}

\subsection{Background}

Through 2008 and early 2009 there was significant concern about the underlying health of the banking system and the banking system's ability to withstand the effects of mounting losses in loan portfolios. After the sharp deterioration in market conditions following Lehman Brother's bankruptcy on September 15, 2008 Congress passed the Emergency Economic Stabilization Act of 2008 (EESA) in early October 2008. The EESA established the Toxic Asset Relief Program (TARP) with \$700 billion with which the U.S. Treasury could purchase assets or equity from distressed financial institutions.

Initially, policymakers intended to use TARP to purchase toxic assets from struggling financial institutions. However, due to the complexity in setting acceptable prices for the securities\footnote{For example, the Federal Reserve and JPMorgan struggled for many months to negotiate appropriate prices for the pool of Bear Stearns mortgage securities JPMorgan agreed to purchase in March 2008 as part of the Maiden Lane I transaction. \citep{Geithner}.}, the logistical challenges of quickly establishing a large purchase program\footnote{The Treasury expected it would take 45 days until the program could begin its purchases. \citep{Geithner}.}, and the relative efficiency of buying assets compared to buying equity, financial regulators and policymakers decided instead to purchase equity from troubled financial institutions. On October 13, 2008 the heads of the U.S. Treasury, Federal Reserve Bank of New York (FRBNY), and the Federal Deposit Insurance Corporation (FDIC) negotiated an aggregate \$125 billion capital injection with the CEOs of nine large financial institutions accompanied with an FDIC guarantee of new bank debt and transaction accounts. The banks could not choose between the two programs and had to accept the capital injection to qualify for the FDIC guarantees, and all nine banks agreed to the capital injections. News of the meeting bolstered equity markets and the S\&P 500 notched its largest daily percent increase, both in points (104.13) and in percentage points (11.58 percent).

This capital injection was the beginning of the Capital Purchase Program (CPP)\footnote{See \citet{Ross2016b} for further discussion of the CPP.} and was funded from the \$700 billion allocated to TARP. The Treasury, Federal Reserve, FDIC and Office of Comptroller of the Currency (OCC) announced the CPP on October 14, with \$125 billion going to the previously negotiated nine financial institutions and another \$125 billion for banks which applied to the program between October and November 2008.\footnote{The application deadline was November 14, 2008. \citep{CPPInstructions}.} \citep{Paulson}. After its final investment in December 2009, the CPP provided about \$205 billion in capital to 707 financial institutions. Figure~\ref{figure1} shows that the initial announcement of the CPP decreased the probability of bankruptcy for the largest banks.

\subsection{Program Description}

Despite the CPP's implementation three months earlier, the fundamental health of the banking system remained in question as President Obama took office in January 2009. On February 10, 2009, the U.S. Treasury, Federal Reserve, FDIC, OCC, and Office of Thrift Supervision (OTS) announced a two part financial stability plan in order to bolster confidence in the banking system and help ensure the banking system continued to intermediate credit. The plan's first component, the Supervisory Capital Assessment Program (SCAP)\footnote{See \citet{Ross2016b} for further discussion of the SCAP.}, was a stress test for the 19 largest bank holding companies (BHCs) in the U.S. which had more than \$100 billion in risk-weighted assets. The plan's second component was the Capital Assistance Program (CAP) a capital backstop available to bank holding companies should regulators deem their capital level insufficient. Initial market reaction was poor -- the Dow fell 382 points after Treasury Secretary Timothy F. Geither's speech introducing the plan. The plan surprised the market by its lack of specificity, but Federal Reserve Chairman Ben Bernanke would later note “with confidence in banks diminishing by the day, I understood the urgency of announcing something.'' \citep{Bernanke}. The involved regulators released the full details on the plan on February 25. \citep{CAPTerms}.


In designing the SCAP and CAP, regulators tried to balance providing the market information on the tail risks surrounding specific institutions while also credibly providing a capital backstop to mitigate the downside risk that would come should, after public disclosure of the stress test results, an institution be in worse condition than expected. The two part financial stability plan worked to provide support to the struggling banking system, create an exit strategy for Treasury investment, and compensate the taxpayer. \citep{GW}.


If the SCAP test indicated a bank needed more capital, the U.S. Treasury offered a capital backstop through purchases of convertible preferred securities and warrants on common equity from BHCs deemed by regulators to have insufficient capital that were also unable raise private capital. Although capital ratios across most U.S. banks had capital beyond their regulatory requirements, in the stressed market conditions market participants paid particular attention to the highest quality capital, common equity, rather than the broader Tier 1 Capital.\footnote{Banks report various metrics of capital adequacy. Market analysts specifically focused on tangible common equity (TCE) as the preferred measure of capital adequacy, which is shareholders' equity less preferred equity and intangible assets. \citep{GW}. TCE is the highest quality form of capital as it is permanently available to absorb losses. TCE is a subset of Tier 1 Capital, which is shareholders' equity plus retained earnings. The SCAP focused on Tier 1 Common Capital, which is Tier 1 capital less preferred shares and non-controlling interests.} Therefore, the convertible preferred securities would be convertible to common equity if either the bank became capital inadequate and required additional capital to meet its supervisory requirements or if needed to ``retain the confidence of investors.'' \citep{WhitePaper}.


The Treasury offered the CAP to all banks and qualifying financial institutions (QFIs) as described by the CPP, unlike the SCAP which was limited to the 19 largest BHCs. BHCs had six months to raise additional capital, but could apply to the CAP immediately after the publication of the SCAP results and delay actual funding for the six months while the firms raised as much private capital as possible. Further, QFIs were allowed to use exchange their existing TARP securities from the CPP with capital from the CAP. 


As was the case with other capital injection programs authorized as part of the EESA, participants of the CAP were subject to the executive compensation requirements in line with the EESA and subject to restrictions on paying quarterly common stock dividends, repurchasing shares, and pursuing cash acquisitions


The public reaction to government ownership of private companies was an important constraint on the various programs undertaken between 2008 and 2009 and the CAP was no exception. In order to mitigate this concern, the Treasury noted “The economy functions better when banking organizations are well managed in the private sector. U.S. government ownership is not an objective of CAP.'' \citep{WhitePaper}. Therefore, authorities designed the CAP to encourage rapid repayment -- primarily through a relatively punitive dividend rate on the preferred security -- and also used a separate trust to administer any capital investments made by the government with the trustees focused on protecting and creating valuing for the government as a shareholder over time. \citep{WhitePaper}.

Qualifying financial institutions (QFIs) were eligible to apply to the CAP. QFIs included BHCs, financial holding companies, insured depository institutions, and savings and loan holding companies, that were organized and operating in the United States, and deemed viable by the appropriate federal banking agency. Financial institutions controlled by foreign entities were ineligible. Holding company applications must have been submitted by January 15, the same date required by the CPP. (Treasury February 25, 2009). Institutions did not need to participate in the CPP in order to be eligible for the CAP. Further, any holding company participating in the CAP must remain a holding company, and was not allowed to terminate its status as a holding company.


In the scenario a QFI used the CAP, the firm would issue mandatorily convertible preferred securities to the U.S. Treasury for between 1 and 2 percent of risk-weighted assets plus an amount necessary to redeem investments made through the CPP or the Targeted Investment Program (TIP), a program established in December 2008 to provide additional support to Citigroup and Bank of America. In the case an institution requires capital beyond this amount, the U.S. Treasury and the relevant federal banking regulator would determine whether the QFI qualified for “exceptional assistance'' in which the Treasury could provide a bank-specific, negotiated agreement. \citep{WhitePaper}.


The CAP provided capital through mandatorily convertible preferred equity securities. The securities carried a 9 percent divided and came with warrants on the QFI's common stock.\footnote{The QFI was required to call a meeting of its shareholders to obtain shareholder consent should the QFI have insufficient authorized common stock to reserve for issuance upon conversion of the CAP preferred shares and/or if stockholder consent is required for issuance under the relevant stock exchange rules. If shareholders delay providing consent the dividend rate on the security increased 20 percent after 6 months. \citep{CAPTerms}.} The security was redeemable at par at any time in the first two years following the capital injection, including any accrued and unpaid dividends. Beyond two years, the security was redeemable but at a penalty rate equal to at least the cost of the conversion option. Importantly, the preferred shares could only be redeemed with the cash proceeds of common stock issuance. (\citet{WhitePaper} and \citet{GW}). The security was senior to any common shareholders and pari passu with other preferred shareholders except other preferred shares' which by their terms rank junior to existing preferred shares[a]. Options to redeem, convert or exercise warrants could done either in part or in full.


At any time within the first seven years the QFI could convert the preferred shares to common shares at a price equal to 90 percent the average closing price for the common stock in the 20 trading days preceding February 9, 2009, the day before the CAP's announcement.\footnote{Subject to to customary anti-dilution adjustments.} \citep{WhitePaper} and \citep{GW}.\footnote{\citet{GW} provide an explicit example: for a preferred share with par of \$100 and an average common stock price of \$5.56 in the days leading up to the CAP announcement, the preferred share would convert into 100/(0.90 x 5.56) = 20 shares of common equity.} The conversion also includes a penalty in the case shareholders delay providing consent: each six months the QFI's shareholders delay consent the conversion price was reduced by 15 percent, up to a maximum reduction of 45 percent the conversion price.


After seven years the preferred shares would convert to common equity at the same price as the conversion option: 90 percent the average closing price of the common stock before the announcement of the the CAP. \citep{WhitePaper} and \citep{GW}.


CAP investments included warrants which allowed the Treasury to purchase shares in the QFI's common shares at any time in the 10 years following the CAP investment. These warrants provided the taxpayer the option to share in the upside of a QFI's recovery. The Treasury could exercise the warrants at a strike price equal to the conversion price that applied to the preferred shares, and the Treasury could exercise warrants up to 20 percent of the ratio of the preferred shares to the common shares. If Treasury were to exercise its warrants it would receive common in the amount equal to 20 percent the par value of the preferred shares. If the bank was no longer public or had insufficient authorized shares, Treasury would exchange the warrants for senior term debt or another appropriate instrument. The Treasury agreed to exercise no voting power with shares obtained through exercise of the warrants. \citep{WhitePaper} and \citep{GW}. be non-voting except for on market terms of similar securities. Upon conversion to common Treasury will become a voting shareholder. \citep{Mofo}. Should the QFI fail to make dividend payments for six quarters the Treasury had the option to appoint two directors, but this right is terminated once the QFI makes dividend payments for four consecutive quarters. \citep{Mofo}.

QFIs could apply to the CAP through their federal banking regulator. If the federal banking regulator agrees that the CAP would be effectively for the QFI, the federal banking regulator would provide Treasury with a recommendation for the firm as well as the application. Treasury would review these and approve or reject the firm's application. Banks were required to submit capital plans as part of the CAP application and were required to use CAP funds to provide credit beyond the levels they would without government assistance. These plans would be made public by the Treasury after the QFI received its CAP investment. Taxpayers were able to monitor the performance of banks receiving capital under the CAP by accessing online monthly reports submitted by the banks to the Treasury showing the status of their lending. (Press Release).


\subsection{Outcomes}

After the Fed released the results of the SCAP, ten firms collectively needed to add \$75 billion to reach their SCAP capital buffer targets. Table~\ref{scapResults} provides the bank specific estimates of capital shortfalls. (FRS May 2009). Nine of those banks were able to fulfill their additional capital needs privately. Ultimately, the SCAP along with the CAP was viewed as a “decisive turning point'' in the financial crisis. \citep{Bernanke}. Only one firm, GMAC (later known as Ally Financial), required public capital to meet its SCAP requirements. GMAC was the only firm to issue securities under the terms of the CAP, however, the firm received funding through the Automotive Industry Financing Program and not through CAP. \citep{OFS} and \citep{GW}. Additionally, the agreement between GMAC and the Treasury provided the Treasury immediate exercise of the warrants at a value of one penny per share. \citep{GW}.



Ultimately, the Treasury did not receive any applications for the CAP, and the program closed on November 9, 2009. The Office of Financial Stability (OFS) viewed this as an indicator of the effectiveness of the SCAP, as well as other government efforts in responding to the financial crisis. \citep{OFS}. Although no QFIs issued securities through the CAP, a handful of firms publicly announced their intention to use the program only to later refrain from applying. Midwest Banc Holdings was in negotiations with the Treasury through October and Citizens Republic Bancorp withdrew its application in November 2009 noting “the lack of activity surrounding CAP.'' \citep{GW}, \citep{Citizens}, and \citep{Midwest}. 

\newpage
\section{Key Design Decisions}

\subsection{The CAP existed as a public backstop should the SCAP stress test reveal capital holes the private sector was unwilling to fill.}

The SCAP conducted a stress test between February 2009 and May 2009, and the regulators conducting the test pledged before the results were released to publicly disclose bank-by-bank results. Market analysts and bank examiners were concerned the results of the test would not be accepted as credible -- the logic suggested that bank examiners would whitewash the test results knowing there was insufficient public capital to recapitalize the banking system as needed. Indeed, Chairman Bernanke notes disagreement between Treasury Secretary Geithner and National Economic Council\footnote{The NEC is a small organization in the Executive Office of the President focused on implementing economic policy.} (NEC) Director Lawrence Summers in the early months of the stress test, a period of considerable uncertainty: “An ongoing question … was what to do if the test revealed a capital hold deeper than could be filled by the remaining TARP funds… [Director Summers] was pessimistic and presumed that, if the stress tests were to be credible, they would have to show catastrophic losses that would overwhelm TARP.


As discussed in \citet{Ross2016a}, market participants ultimately viewed the SCAP as credible. The test scenario was particularly adverse, projecting a two-year commercial loan-loss rate in excess of 9 percent -- higher than any two-year period since 1920. Chairman Bernanke credits the CAP capital backstop as an important component of the test's perceived credibility:
The availability of backstop capital gave the regulators the right incentives: Without it, we might have been suspected of going easy on weaker banks, for fear of inducing runs. With the backstop, investors could see that we had every reason to be rough, to ensure that troubled banks would be forced to take all the capital they needed to remain stable. \citep{Bernanke}.
The test made clear, however, that the level of capital shortfall could have been replenished with TARP funds. It is unclear how the CAP capital backstop would have been used had the capital shortfall exceeded the remaining TARP funds.

\subsection{The CAP securities carried a higher dividend than the CPP, but included the option to convert to equity at any time (and mandatorily after seven years.}

The CPP carried a 5 percent dividend for the first five years which increased to 9 percent thereafter, whereas the CAP carried a 9 percent dividend from the start. Unlike the CPP, however, the CAP allowed conversion to common equity at any point -- at a 10 percent discount to the share price in the 20 days leading up to the CAP's announcement. It is clear this conversion option was a key component of the program, as the CAP allowed banks to exchange their CPP shares to CAP shares beyond their maximum injection in terms of percent of risk-weighted assets. \citep{GW}.

While the conversion option may have been worth the increased dividend to some banks, there was possibly an expectation of implicit convertibility in the CPP shares -- Citi converted its CPP shares in the weeks immediately following the CAP announcement. Therefore, many banks may have felt the increased dividend was not worth the explicit convertibility option.

\subsection{The CAP allowed firms to convert their preferred CAP securities to common equity.}

The CAP allowed conversion of the preferred shares to common equity at any point, although after seven years the conversion was mandatory. The predecessor program, CPP, had no such feature. Conversion was available to QFIs at 90 percent of the equity price of the bank in the days leading up to the February 23, 2009 announcement of the CAP. This feature of the program addressed the concern that market analysts had surrounding bank's capital adequacy; while most banks had sufficient Tier 1 Capital, many analysts felt there was insufficient capital of the highest form. Therefore, the conversion option allowed banks to convert the preferred shares to common, thereby changing the composition of their Tier 1 capital and boosting their TCE because TCE is higher quality capital available permanently to absorb losses with no debt-like characteristics.

\subsection{Banks, not the government, initiated conversion to common equity.}

The program allowed banks to initiate the conversion to common either due to additional losses or to bolster confidence in the bank. Regulators designed the CAP to reduce the likelihood Treasury became an active voting shareholder due to political concerns about nationalization of the banking system and the government's lack of expertise in running financial institutions. Under certain circumstances -- if the bank missed dividend payments for six quarters, for example -- the Treasury would take a more active role, but the program was designed to restrain Treasury's involvement as a shareholder.

\subsection{Foreign financial institutions were ineligible for the CAP.}

The CAP, consistently with the SCAP, used the same definition of QFI as defined for the purposes of the CPP as unveiled in the fall of 2008. Notably, this excluded foreign institutions and U.S. branches or agencies of foreign institutions. This is largely due to the fact that foreign bank branches and agencies have no capital of their own and are subject to a different set of regulatory requirements than depository institutions in the US. Therefore, it is not possible to stress test their capital adequacy.\footnote{For further discussion of Federal Reserve regulation of foreign institutions, see \url{https://www.newyorkfed.org/aboutthefed/fedpoint/fed26.html}.}

\section{Evaluation}

Comparison to the CPP. The CAP followed the CPP by four months and affected a similar set of financial institutions. The dividend rate was an important difference between the two program's designs as The CAP securities carried a 9 percent dividend, compared to the CPP's 5 percent dividend for the first five years which subsequently ratched to 9 percent afterwards. The CPP also provided no time deadline for redemption\footnote{As initially designed in late 2008, the CPP required firms to redeem through equity issuance within three years. The American Recovery and Reinvestment Act of 2009 (ARRA, also called “The Stimulus”) removed this time requirement. \citep{GW}.} and also incentivized repayment by cancelling a portion of Treasury's warrants if the QFI issued equity. \citep{GW}. The CPP had no conversion to common option, whereas CAP did. Why did the U.S. banking system need the CAP as the CPP focused on the the same issue for the same set of institutions, especially considering the higher dividend rate required on CAP securities? The deadline for QFIs to apply to the CPP was November 14, 2008. \citep{CPPFaq}. QFIs with questionable capital adequacy after that date would need the CAP for assistance.

It was not the case that all QFIs had shored up their balance sheets after November 14, and therefore had no use for the CAP. It is unclear why the CAP was not used further. \citet{GW} use a contingent claims framework to value the CAP securities and find the CAP securities had a net value of approximately 30 percent of the capital invested for the banks participating with the maximum investment relative to risk-weighted assets. They also perform an event study of the banks included in the SCAP and find close correlation between the announcement of the terms of the CAP and abnormal price returns for the banks. \citep{GW}.

A handful of hypotheses have emerged to address why the CAP -- although valuable to banks -- went unused.

First, it is possible the TARP limits of executive compensation and hiring of foreign workers could put the firm at a competitive disadvantage, or that partial public ownership could reduce a firm's ability to continue profitable business lines in the face of public resistance. For example, Citigroup sold Phibro, a profitable energy trading subsidiary, to avoid a particularly large contractual bonus payment of \$100 million to a trader. \citep{GW} and \citep{Dash}. TARP limits also required senior management reviews, which could have led to agency issues in which management -- concerned about compensation limits and reviews -- refrained from joining CAP. \citet{GW} find that this may have been a constraint for many other TARP programs, including the CPP, and therefore cannot explain the disuse of the CAP.

Second, participation in the CAP may have been discouraged by fear of stigma. \citet{GW}'s analysis finds the most valuable aspect of the CAP is the optionality of converting the preferred shares to common equity at a favorable ratio. QFIs -- with material nonpublic information to the downside -- that would benefit most from the CAP would fear that participation in the CAP would give a negative signal to the market, thereby increasing the cost of raising private capital in the future. As \citet{GW} say, “[a] value-maximizing firm may pass up an opportunity to participate in a positive [net present value] program if doing so lowers its cost of raising private capital.''

Third, the terms of the preferred CAP shares required repayment with the cash proceeds of common equity issuance. Simplified, the CAP provided QFIs four options: (1) do not participate and forego the capital investment, (2) participate and redeem the CAP preferred shares with cash from common equity issuance, (3) participate without the ability to repay in the medium term (less than seven years) and pay the 9 percent dividend before ultimately exiting the program before mandatory conversion, or (4) participate and convert to common either before or exactly at seven years. Only options (3) and (4) would be available to a QFI which participated in the CAP but faced a high cost of common equity issuance, and the QFI must decide between paying a high dividend or dilute common shareholders.

Fourth, \citet{Caballero} sees dilution as a driver concern for many QFIs. The KBW Bank Index, an index of the 24 largest U.S. publicly traded U.S. banks, had fallen 64 percent from January 2008, implying a CAP conversion price of about 32 percent the prior valuation. Figure~\ref{figure2} shows the pressure bank stocks faced through 2008 and 2009. CAP capital provided at an at January and early February 2009 valuations could prove very dilutive. Caballero instead suggested the creation of an equity price guarantee five years later for all new private equity injections. Banks requiring additional capital -- either those identified through the SCAP or otherwise -- could raise private capital in the near term with a government guarantee on equity prices five years later. The proposal aimed at removing the possibility of a dilution extreme enough to be rejected by management, and instead “transforms heavily discounted equity into a Treasury bond with a call option on the upside of the bank.''

Fifth, \citet{GW} see the ability to convert to common as one of the most enticing aspects of the CAP, as the CPP had no such feature: “this appears to be the intention of regulators because they included a provision allowing banks to use CAP funds to repay CPP funds.'' \citep{GW}. The CAP provided banks the option to exchange their 5 percent CPP preferred shares with the 9 percent CAP preferred shares with a conversion option. If the option to convert exceed the 4 percent difference in the preferred dividend, then QFIs may have chosen to use the CAP. However, it is likely QFIs felt CPP preferred shares were implicitly negotiate in the scenario tangible common equity fell to sufficiently low levels. Notably, Citigroup boosted its tangible common equity by converting its CPP preferred shares to common equity on February 27 after negotiations with Treasury. \citet{GW} note there were also reports of similar agreements between the Treasury and other banks.\footnote{For example, ``US May Convert Banks' Bailouts to Equity Share,'' \citep{Andrews} and \citep{GW}.} With this precedent of conversion\footnote{\citet{GW} also note PNC Financial announced it had decided not to convert its CPP shares, implying it had indeed considering converting them.}, some QFIs were likely unwilling to pay the steeper dividend rate in exchange for the explicit option to convert.

Finally, the firms found to have inadequate capital by the SCAP had until June 8, 2009 to submit their capital raising plans. Average equity prices of the 18 BHCs tested in the SCAP had increased on average 65 percent from the low in February. \citet{GW}'s valuation of the CAP falls from 30 percent of the net capital invested in the tested banks to about 3 percent on June 5 as a result of higher equity prices.\footnote{However, \citet{GW} point out some banks still had much to gain from the CAP, mainly Citi and Keycorp.} In evaluation the CAP designed, the note that the CAP had a sort of automatic exit strategy: “As bank stock prices increased, the net value to banks of the CAP securities decreased, creating an almost automatic termination of the program.'' \citep{GW}.

Although the CAP ultimately went unused, many have pointed to the CAP as a useful tool in conjunction with the SCAP.

\newpage
\phantomsection

\addcontentsline{toc}{section}{References}

\nocite{*}
\bibliography{\jobname}

\section{Appendix A - List of Resources}

\subsection{Summary of Program}

\begin{itemize}

\item
\ul{Treasury White Paper: The Capital Assistance Program and its Role in the Financial Stability Plan}, U.S. Treasury, February 2009 -- \emph{Treasury white paper describing how the CAP fits within the broader financial stability plan, the contingent capital framework, and the program's specific design elements.} \url{https://www.treasury.gov/press-center/press-releases/Pages/tg40.aspx}
\end{itemize}

\subsection{Implementation Documents}
\begin{itemize}
\item
\ul{Term
 Sheet for Capital Assistance Program}, U.S. Treasury -- \emph{Treasury
 document discussing terms of investments made via the CAP.} \url{http://www.treasury.gov/press-center/press-releases/Documents/tg40_captermsheet.pdf}
\item
\ul{The
 Supervisory Capital Assessment Program: Design and Implementation},
 Board of Governors of the Federal Reserve System, April 24, 2009 -- \emph{Federal Reserve document outlining design details of the SCAP.} \url{http://www.federalreserve.gov/bankinforeg/bcreg20090424a1.pdf}
\end{itemize}

\subsection{Legal/Regulatory Guidance}

\begin{itemize}
\item
\ul{Recovery Plan's Retroactive Restrictions and Say-on-Pay}, Morrison Foerster, March 2009 -- \emph{The American Recovery and Reinvestment Act of 2009 changed executive compensation restrictions on firms participating in EESA programs; this document outlines the relevant changes.} \url{http://media.mofo.com/files/uploads/Images/090302NewEra.pdf}
\end{itemize}

\subsection{Press Releases/Announcements}

\begin{itemize}
\item
\ul{U.S. Treasury Releases Terms of Capital Assistance Program}, U.S. Treasury, February 25, 2009 -- \emph{Treasury press release describing how the Treasury and Federal banking agencies would test large bank holding companies with the SCAP and how the CAP would be used together with the SCAP.} \url{http://www.federalreserve.gov/newsevents/press/bcreg/bcreg20090507a1.pdf}
\item
\ul{Treasury Announcement Regarding the Capital Assistance Program}, November 9, 2009 -- \emph{Press release announcing the closure of the CAP after making no investments.} \url{https://www.treasury.gov/press-center/press-releases/Pages/tg359.aspx}
\item
\ul{SCAP
 Results}, Federal Reserve, May 7, 2009 -- \emph{Press release which
 announces the results of the SCAP.} \url{http://www.federalreserve.gov/newsevents/press/bcreg/20090507a.htm}
\item
\ul{Statement
 by Timothy F. Geithner U. S. Secretary of the Treasury before the
 Senate Banking Committee}, May 20, 2009 -- \emph{Secretary Geithner
 discusses the initial impact of and market response to the SCAP's
 results.} \url{https://www.treasury.gov/press-center/press-releases/Pages/tg139.aspx}
\end{itemize}

\subsection{Media Stories}

\begin{itemize}
\item
\ul{U.S. May Convert Banks' Bailouts to Equity Share}, New York Times, April 19,
 2009 -- \emph{Article discussion the possibility of banks converting CPP shares to common equity.} \url{http://www.nytimes.com/2009/04/20/business/20bailout.html}
\item
\ul{Citigroup Sheds Energy Unit and Its \$100 Million Trader}, New York Times, October 9,
 2009 -- \emph{Article discussion sale of Philbro by Citigroup.} \url{http://www.nytimes.com/2009/10/10/business/10citi.html}
\end{itemize}

\subsection{Key Academic Papers}

\begin{itemize}
\item
\item
\ul{A (Mostly) Private Capital Assistance Programme
(CAP)},
Richard Caballero, 2009 -- \emph{Paper describing an alternative to the CAP which set a government guaranteed floor on bank stock prices.} \url{http://voxeu.org/article/recapitalising-banks-caballero-plan}
\ul{Valuing the Treasury's Capital Assistance Program},
Paul Glasserman and Zhenyu Wang, 2011 -- \emph{Paper
which finds CAP to be very valuable to banks, with a discussion of why banks ultimately did not participate in the program.} \url{http://papers.ssrn.com/sol3/papers.cfm?abstract_id=1525640}
\end{itemize}

\subsection{Reports/Assessments}

\begin{itemize}
\item
\ul{Troubled
 Asset Relief Program: Two Year Retrospective}, Office of Financial
 Stability, October 2010 -- \emph{Office of Financial Stability report
 discussing the program and its outcomes in the context of the wider
 swath of TARP.} \url{http://www.treasury.gov/press-center/news/Documents/TARP\%20Two\%20Year\%20Retrospective_10\%2005\%2010_transmittal\%20letter.pdf}
\end{itemize}

\section{Appendix B - Road Map}

The following is a list of the key design decisions that will likely have to be made in implementing a program similar to the Capital Assistance Program (CAP), a capital backstop program available to large bank holding companies deemed to have insufficient capital following a stress test.

\subsection{Key Questions}

\begin{outline}[enumerate]

\1 Which agency or agencies have the authority and expertise to provide the capital backstop?
\2 What is the basis of this authority?
\2 What particular elements/terms must be satisfied to fit within the authority?
\2 After designing, have all required elements been satisfied?
\2 Is any additional authority required in order to provide a capital backstop?
\2 How long should firms be allowed to seek private capital before turning to the public backstop?
\1 How should a public capital backstop be structured?
\2 What sort of security should the public capital be provided through?
\2 Should economic conditions worsen, can the public capital convert into common equity?
\3 If so, should the securities convert to common at a discount or at face value?
\2 How can the backstop be structured to compel firms to first raise private capital and use the public capital as a less preferred option?
\2 Does the backstop come with a dividend? If so, what is the right balance between providing capital to firms that otherwise cannot raise capital but is also sufficiently punitive that firms work to replace it with private capital quickly?
\2 Is there mandatory conversion to common after a time period? If so, after how long?
\2 How does the taxpayer participate in the potential future profitability of the involved firms? Does the public receive warrants, for example?
\2 How does the public exit its investment? Over what time frame? 
\2 How can participating financial institutions redeem their capital injections? With cash proceeds from equity issuance only, as in the CAP? 
\1 To what extent does the government participate ?
\2 To what extend does the public influence management decision making?
\2 What other constraints will firms using public capital face? (E.g. executive compensation caps, restrictions on common stock dividends, buybacks and cash acquisitions, etc.)
\2 Are there sufficient authorized shares to meet the capital backstop's requirements?
\2 Does the capital injection trigger any poison pill or covenants?
\2 What is the relationship between the capital injection's preferred shares and existing preferred shares? 
\1 Which firms are eligible for the capital backstop? 
\2 Are foreign institutions eligible?
\2 What tests are conducted to determine capital adequacy and the amount of support the public should provide? (E.g., is there a stress test?)
\2 What metric or measure should regulators target to assess capital adequacy?
\3 Should the test focus on Tier 1 capital, Tier 1 Common capital, tangible common equity, a combination of these or something else?
\4 For example, should preferred equity, goodwill and intangible assets be included in the equity component?
\4 Should the denominator be based on risk-weighted assets, tangible assets or something else?

\end{outline}

\subsection{Implementation Steps}

\begin{enumerate}

\item Develop the description of the capital backstop, including legal authority, purpose, firm eligibility, a general timeline, et cetera and seek input from industry and other stakeholders.
\item If necessary, seek approval for the program, funding et cetera.
\item Produce term sheet for the program.\footnote{CAP Example Term Sheet: \newline \url{https://www.treasury.gov/press-center/press-releases/Documents/tg40_captermsheet.pdf}}
\item Develop application instructions for completing the documentation necessary to participate in the capital back stop.
\item Produce capital adequacy targets with which to judge applications.
\item Find institution specific capital adequacy using supervisors and firms own' capital adequacy estimates. 
\item Compare supervisors' capital adequacy estimates with firms' own estimates and reconcile differences.
\item Provide capital to firms with inadequate capital. 

\end{enumerate}

\newpage
\section{Figures and Tables}

\begin{figure}[h]
\caption{CPP and SCAP Eased Market Pressures, CAP Did Not.}\label{figure1}
\centering
\includegraphics[width=\textwidth]{CDS.pdf}
\raggedright
\footnotesize Source: Bloomberg.
\end{figure}

\begin{table}[htbp]
\setlength\LTleft\fill
\setlength\LTright{0pt}
\begin{longtable}[l]{@{\extracolsep{\fill}}@{}ll@{}ll@{}}
\caption{SCAP Results}\label{scapResults}\\
\toprule
\textbf{Firm} & \textbf{Additional Capital Required} &\tabularnewline
\midrule
\endhead
Bank of America & \$33.9 billion & ~\tabularnewline
Wells Fargo & \$13.7 billion &\tabularnewline
GMAC & \$11.5 billion & ~\tabularnewline
Citigroup & \$5.5 billion &\tabularnewline
Regions Financial Corp. & \$2.5 billion & \tabularnewline
SunTrust Banks & \$2.2 billion &\tabularnewline
Morgan Stanley & \$1.8 billion & \tabularnewline
KeyCorp & \$1.8 billion &\tabularnewline
Fifth Third Bank & \$1.1 billion & \tabularnewline
PNC & \$0.6 billion &\tabularnewline
American Express & Adequate & \tabularnewline
Bank of New York Mellon & Adequate &\tabularnewline
BB\&T & Adequate & ~\tabularnewline
Capital One & Adequate &\tabularnewline
Goldman Sachs & Adequate & \tabularnewline
JP Morgan Chase & Adequate &\tabularnewline
MetLife & Adequate & ~\tabularnewline
State Street & Adequate &\tabularnewline
U.S. Bancorp & Adequate & \tabularnewline
\bottomrule
\multicolumn{3}{l}{\footnotesize Source: Federal Reserve (May 7, 2009).}
\end{longtable}
\end{table}

\begin{figure}[h]
\caption{Bank Valuations Around Time of CAP Announcement}\label{figure2}
\centering
\includegraphics[width=\textwidth]{bank_index.pdf}
\raggedright
\footnotesize Source: Bloomberg.
\end{figure}

\end{document}